\chapterimage{chapter_head_2.pdf} % Chapter heading image

\chapter{Introduction}

\section{What is Zim Code?}

Zim Code is a nonprofit coding school that was founded by Alvin Chitena in 2016. Our mission at Zim Code is to bringing coding to the forefront of youth education and empowerment in Zimbabwe. Code is the language of the present and the future, and so we believe that it is necessary for everyone to have the opportunity to learn it. Zim Code is made of young and dedicated Zimbabweans from all walks of life who are passionate about coding, education, entrepreneurship and philanthropy. 

Given the vast socio-economic issues that plague Zimbabwe, we see it necessary to remove the many barriers that people may face when trying to learn how to code, namely quality information and resources. This course, the Introduction to Programming: Python (12-weeks) is one of the many initiatives we are spearheading to further our mission.

To find out more about Zim Code, visit our website: www.zimcode.org

\section{Purpose of the textbook}

This text is designed to cater specifically to high school students in Zimbabwe and other African
countries whose education system is based on the GCE A level curriculum. The idea is to
provide a low barrier text in writing that is easy to understand for African high school students with very little assumptions made about their background. 

A wide range of relatable examples are used throughout the text to provide analogies and help students grasp the abstract concepts in Computer Science and Programming. This is done with the hope that students may begin to appreciate how Computer Scientists and Software Engineers solve problems and how programming skills can be applied in the real world.

In addition to learning the basics of coding, students will also inevitably learn and improve in other subject areas. The link between Computer Science and Mathematics is undeniable and in fact, fundamental. While it is easy to fall into the temptation of providing mostly computational and mathematical examples, we tried our best to include a wide range of examples appealing to 
all three major fields of study: Science, Commerce and Humanities. 

This course will give a 12-week introduction to programming in Python which is one of the most accessible languages to beginners due to its similarity to written English and white space delimiting. White space delimiting is the use of indentations and line 
breaks rather than brackets to group code blocks e.g. in a function/loop. Python is also a very powerful tool in computation (The NumPy package is used in various fields for numerical computation) and web design (Instagram is written in Django/Flask, Python frameworks) 
with a very good support and development network. It is also easy to install and is often preinstalled on most operating systems (unfortunately, not on Windows).

An important part of this text is that it should be fun and interactive to the students. This text is "minimal" in that to benefit the most out of the content, the reader has to go through all the readings, examples and exercises. It involves live demoes and even tries to introduce real world applications early on. Examples are chosen from real world problems so that it becomes immediately apparent to the students how programming applies to real life.
Students will also learn a lot of broader concepts in programming such as abstraction, functional programming, and algorithmic thinking.

This course is designed to be open, fun and interactive. We encourage deviating from the standard lecture model because the best way to learn programming is by actually doing it. The style will be a combination of lecture and lab type classes where a concept is introduced for 
a few minutes, the students get to use it and then reflection on the use of the concept will hopefully bring to light its value and clear any misconceptions. 

While we provide a detailed lesson plan to teachers in addition to this textbook, experimentation and flexibility is encouraged so that the most can be made from every class. If something does not work particularly well, change it and if the teacher has better ways of teaching the class in their context, they should. 

We would like feedback on what worked and what didn’t, if you have any, email info@zimcode.org
